%==============================================================================
% document template for a LaTeX file
% Created : 22.11.2006 by Romain de Wolff
% Modif : 8 octobre 2007, respect standart HEIG-VD pour travaux de diplômes
%==============================================================================
% configurations du document
\documentclass[a4paper, 11pt]{article}
\usepackage[utf8]{inputenc}
\usepackage[cyr]{aeguill}
\usepackage[frenchb]{babel}               % \og et \fg pour les guillemets
\usepackage{url}                          % pour inclure des URLs
\usepackage{color}                        % utilisé pour le code source
\usepackage{listings}                     % pour inclure le code source
\usepackage{geometry}                     % pour les marges
\usepackage{algorithmic}                  % pour présenter les algos
\usepackage{algorithm}                    % pour présenter les algos
\usepackage{graphicx}                     % pour inclure des images
\usepackage{fancyvrb}                     % pour avoir du verbatim encadré
\usepackage{url}                          % pour inclure des URLs
\usepackage{geometry}                     % See geometry.pdf to learn the layout options
\usepackage{moreverb}					  % permet d'encadrer le verbatim à l'aide de la commande \begin{boxedverbatim}
\usepackage{longtable}

\geometry{a4paper}                        % ... or a4paper or a5paper or ... 
%\geometry{landscape}                     % Activate for for rotated page geometry
\usepackage[parfill]{parskip}             % Activate to begin paragraphs with an empty line rather than an indent
%\usepackage{amssymb}
%\usepackage{epstopdf}                    % Pour créer des références divers (webographie, etc) 
\usepackage{bibtopic}
\geometry{ hmargin=2.8cm, vmargin=2.5cm } % marges
%==============================================================================
% debut macro 
% permet de mettre en gros la premiere lettre d'un paragraphe
\font\capfont=cmbx12 at 44.87 pt % or yinit, or...?
\newbox\capbox \newcount\capl \def\a{A}
\def\docappar{\medbreak\noindent\setbox\capbox\hbox{%
\capfont\a\hskip0.15em}\hangindent=\wd\capbox%
\capl=\ht\capbox\divide\capl by\baselineskip\advance\capl by1%
\hangafter=-\capl%
\hbox{\vbox to8pt{\hbox to0pt{\hss\box\capbox}\vss}}}
\def\cappar{\afterassignment\docappar\noexpand\let\a }
% fin macro
%==============================================================================

%==============================================================================
% première page
\title{\includegraphics[width=100px]{HEIG-VD.jpg} \\ \vspace{4cm}
\small{PPE - Présentation personnelle} \\ \vspace{2cm}
\huge{WebDAV} \\ \vspace{1cm} 
Web-based Distributed Authoring and Versioning \\ 
\small{Un protocole réseau pour l'édition collaborative sur le web}} 
\vspace{2cm}
\author{Romain \bsc{de
Wolff} \\ IL2008 \\ \vspace{2cm} \\ Professeurs Mme. Laura Elena Raileanu \\ et Mme Nastaran Fatemi \vspace{2cm} 
}
\date{Lausanne, le 8 octobre 2007}  % Activate to display a given date or no date
\pagebreak{}
\begin{document}
\maketitle
\thispagestyle{empty} % enlève le numéro de page sur la page de titre (uniquement)
\newpage
%==============================================================================
% configuration des numérotations de pages (chiffres romains)
\pagenumbering{roman} \setcounter{page}{1} 

% Configuration de la distance d'interligne 
{\setlength{\baselineskip}{1.2\baselineskip}
% Configureation de la distance entre les paragraphes
\parskip=12pt
%==============================================================================
%==============================================================================
\section*{Objectifs} 
%==============================================================================

%==============================================================================
\section*{Démarches adoptées} 
%==============================================================================

%==============================================================================
\section*{Réponses aux questions} 
%==============================================================================

%==============================================================================
\section*{Analyse des résultats obtenus} 
%==============================================================================

%==============================================================================
\section*{Conclusion personnelle} 
%==============================================================================


%==============================================================================
\end{document}  
