% --------------------------------------------------------------------------
% document template for a LaTeX file
% Created : 22.11.2006 by Romain de Wolff
% Modif : 8 octobre 2007, respect standart HEIG-VD pour travaux de diplômes
% --------------------------------------------------------------------------
\documentclass[a4paper, 11pt]{article}
\usepackage[utf8]{inputenc}
\usepackage[cyr]{aeguill}
\usepackage[frenchb]{babel}               % \og et \fg pour les guillemets
\usepackage{url}                          % pour inclure des URLs
\usepackage{color}                        % utilisé pour le code source
\usepackage{listings}                     % pour inclure le code source
\usepackage{geometry}                     % pour les marges
\usepackage{algorithmic}                  % pour présenter les algos
\usepackage{algorithm}                    % pour présenter les algos
\usepackage{graphicx}                     % pour inclure des images
\usepackage{fancyvrb}                     % pour avoir du verbatim encadré
\usepackage{url}                          % pour inclure des URLs
\usepackage{geometry}                     % See geometry.pdf to learn the layout options. There are lots.
\geometry{a4paper}                        % ... or a4paper or a5paper or ... 
%\geometry{landscape}                     % Activate for for rotated page geometry
\usepackage[parfill]{parskip}            % Activate to begin paragraphs with an empty line rather than an indent
\usepackage{longtable}
%\usepackage{amssymb}
%\usepackage{epstopdf}
\geometry{ hmargin=1.8cm, vmargin=2.5cm } % marges

% ------------------- TITRE ------------------------
\title{%\includegraphics[width=100px]{heig-vd-logo.pdf} \\ \vspace{4cm}
\small{Laboratoire} \\ ASI : Applications et Services Internet \\ \vspace{2cm}
\huge{Serveur Web Multi-Threadé en Java} \vspace{2cm}} \author{Romain \bsc{de
Wolff} \\ IL2008 \\ \vspace{2cm} \\ Professeur M. Christian Buchs \vspace{2cm} }
\date{Lausanne, le 8 octobre 2007}  % Activate to display a given date or no date
\pagebreak{}
\begin{document}
\maketitle
% ---------------------- PAGE BREAK ---------------------------
\newpage
% ----------------- TABLE DES MATIERES ------------------------
\tableofcontents
%\newpage
% pour faire les 6 premier chapitres manquant et pas a rendre
\section{bidon}
\section{bidon}
\section{bidon}
\section{bidon}
\section{bidon}

% TODO insérer le nombre de pages JUSTE comment avant

\newpage
%%%%%%%%%%%%%%%%%%%%%%%%%%%%%%%%%%%%%%%%%%%%%%%%%%%%%%%%%%%%%%%%%%%%%%%%%%%%%%%
\section*{2ème phase}

\section{Développeur}

Le développeur est Cyril Maulini.

\section{Tests de validation} 

\subsection{Introduction}
blabla

\subsection{Test 1: lundi 15 octobre 2007}

\subsubsection{Résultats des tests}
\begin{longtable}{ c | c | p{14cm} | c}
CT & Sp & Résultat attendu & OK \\
\hline
\hline
1 & 1 & Le serveur démarre et est accessible par plusieurs navigateurs en même temps & OK \\
\hline
2 & 2 & En même temps que le fichier volumineux se télécharge, le serveur doit répondre et renvoyer la page d'accueil dans la deuxième fenêtre & ERR \\
\hline
3 & 3 & Le serveur affiche la page de bienvenue dans le navigateur au format HTML & OK \\
\hline
4 & 4 & Uniquement le fichier demandé doit être retourné et la connexion terminée & OK \\
\hline
5 & 5 & Le serveur doit afficher des messages d'erreur quand le port est plus petit que 1024 ou plus grand que 65535. Sinon le serveur doit démarrer correctement et être accessible via le navigateur internet & OK \\
\hline
6 & 6 & Le compteur de \emph{threads} du processus serveur devrait augmenter en fonction du nombre de connexions que vous effectuez à l'aide du navigateur & OK \\

7 & 7 & Un message affiché entre des balises HTML doit être retourné, informant que la méthode n'est pas implémentée & OK \\
\hline
8 & 8 & Le navigateur doit afficher les pages correctement & OK \\
\hline
9 & 9 & Le serveur doit se terminer & OK \\
\hline
10 & 10 & La réponse du serveur doit comporter les champs suivant, affiché ligne par ligne : \texttt{Date}, \texttt{Server}, \texttt{Content-Length} et \texttt{Content-Type} & ERR \\
\hline
11 & 11 & L'analyse des trames dans Wireshark montre clairement qu'une connexion TCP par objet est créée & OK \\
\hline
12 & 12 & La fenêtre dans laquelle le serveur à été lancé doit afficher les informations qui sont envoyées au client exactement de la même manière que lors d'une connexion telnet & ERR \\
\hline
13 & 13 & Le contenu du fichier \texttt{index.html} doit être retourné & ERR \\
\hline
14 & 14 & Le fichier doit s'afficher dans le navigateur et son contenu être identique & ERR \\
\hline
15 & 15 & Le serveur doit retourner un message HTML personnalisé indiquant que la page n'existe pas. Cette page \underline{ne doit pas figurer} dans un fichier HTML contenu dans le même ou un sous répertoire que le serveur & OK \\
\hline
16 & 16 & La requête est affichée à l'écran suivi de l'en tête du message. Chaque information est affichée sur une nouvelle ligne & OK \\
\hline
17 & 17 & Le serveur doit afficher un message au format HTML. Vérifier qu'il n'existe pas de fichier HTML dans le dossier du serveur qui contient le texte affiché à l'écran & OK \\
\hline
18 & 18 & - & OK \\
\hline
19 & 19 & L'en-tête doit comporter le champs \texttt{Content-Type : text/html} & OK \\
\hline
20 & 20 & Le champs \texttt{Content-Type} doit retourner, dans l'ordre : \texttt{text/html}, \texttt{image/jpeg}, \texttt{image/gif} & OK \\
\hline
21 & 21 & L'en-tête doit comporter le champs \texttt{Content-Type : application/octet-stream} & OK \\
\hline
22 & 22 & Le fichier par défaut \texttt{index.html} doit être retourné & ERR \\
\hline
23 & 23 & Le navigateur va se retrouver à l'adresse \texttt{http://www.heig-vd.ch} & OK \\ 
\hline
24 & 24 & Les fichiers sources du serveur doivent y figurer et être visible en mode texte directement dans le navigateur & OK \\
\hline % fin de la table
\end{longtable}


\subsubsection{Corrections à apporter}

\begin{longtable}{ c | c | p{15.4cm}}
N & CT & Remarque \\
\hline
\hline
1 & 2 & Après avoir placé un fichier de 300 MB à télécharger, le serveur me propose de télécharger un fichier de taille erroné. Le fichier proposé fait \textbf{66 GO} d'après le serveur. Un fichier comportant un nom avec des espaces ne fonctionnera pas non plus. \\
\hline
2 & 10 & Les champs \texttt{Date} et \texttt{Server} sont manquant. Le champs \texttt{Date} doit comporter la date et l'heure et le champs \texttt{Server} une information sur le nom et la version du serveur \\
\hline
3 & 12 & La console du serveur n'affiche pas les sorties HTML comme il le fait lors d'une connexion telnet \\
\hline
4 & 13 & Le serveur retourne une erreur alors qu'il devrait traiter la méthode \texttt{POST} comme la méthode \texttt{GET} \\
\hline
5 & 14 & Le serveur à défini un sous-répertoire pour y mettre les sources, parfait. Par contre son contenu est différent, une dernière ligne est affichée contenant \texttt{Content-Type : text/html} et \texttt{Content-Length : X} en trop \\
\hline
6 & 22 & Le serveur ne renvoie pas le document par défaut \texttt{index.html} \\
\hline % fin de la table
\end{longtable}


\subsubsection{Tests des corrections}

\begin{longtable}{ c | c | p{14 cm}|c}
	N & CT & Remarque & OK\\
	\hline
	\hline
	1 & 2 & Un fichier de grande taille est maintenant téléchargeable correctement & OK \\
	\hline
	2 & 10 & Les champs sont maintenant présent et affiché correctement & OK \\
	\hline
	3 & 12 &  Le contenu des fichiers HTML n'est toujours pas visible dans la console du serveur & (*) \\
	\hline
	4 & 13 & La méthode \texttt{POST} est bien reconnue comme un \texttt{GET} & OK \\
	\hline
	5 & 14 & Problème non résolu. Cf. erratum & (*) \\
	\hline
	6 & 22 & Le serveur renvoie le document par défaut \texttt{index.htm} à l'aide de la commande \texttt{GET HTTP1.0} & OK \\
	\hline % fin de la table
\end{longtable}

(*) Se référer à la section \texttt{Erratum} pour plus d'information.

\subsubsection{Tests de non-régression}

Les tests de non-régressions se sont déroulé sans aucun problème.

\subsubsection{Analyse et synthèse}

	Le déroulement du projet c'est passé en plusieurs étapes dont voici une description plus complète.
	
	- Réalisation des tests: chaque élément de la liste est testé soigneusement. En fonction des résultats obtenus, on valide, ou pas, le résultat \\
	
	- Commentaires: les tests invalides sont décrit de manière plus précise afin que le développeur puisse à son tour le reproduire et le corriger \\
	
	A ce moment, les résultats intermédiaires sont renvoyés au développer. Celui-ci nous retourne les corrections que l'on va valider. \\
	
	- Tests des corrections: les tests invalides sont testé encore une fois. 
	
	- Test de régression : Le premier cycle de test touche à sa fin.
	
\subsection{Conclusion}

Un seul cycle de tests fut nécessaire.

\section{Réponses aux questions}

	Q: Le serveur établis-il des connections persistantes ou non-persistantes? \\
	R: 
	
	Q:
	R:
	
	Q: Des connections parallèles sont utilisées par le serveur? \\
	R: 
	
\section{Erratum}

Les spécifications contiennent des erreurs. C'est lors de la discussion avec le développeur ayant reçu la même donnée que nous, que nous nous en sommes rendu compte. 



\section{Annexes}

Code sources % TODO code source du dév ?






%%%%%%%%%%%%%%%%%%%%%%%%%%%%%%%%%%%%%%%%%%%%%%%%%%%%%%%%%%%%%%%%%%%%%%%%%%%%%%%
\section{Conclusion}


%%%%%%%%%%%%%%%%%%%%%%%%%%%%%%%%%%%%%%%%%%%%%%%%%%%%%%%%%%%%%%%%%%%%%%%%%%%%%%%	
\end{document}  
