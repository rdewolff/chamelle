% --------------------------------------------------------------------------
% document template for a LaTeX file
% Created : 22.11.2006 by Romain de Wolff
% Modif : 8 octobre 2007, respect standart HEIG-VD pour travaux de diplômes
% --------------------------------------------------------------------------
\documentclass[a4paper, 11pt]{article}
\usepackage[utf8]{inputenc}
\usepackage[cyr]{aeguill}
\usepackage[frenchb]{babel}               % \og et \fg pour les guillemets
\usepackage{url}                          % pour inclure des URLs
\usepackage{color}                        % utilisé pour le code source
\usepackage{listings}                     % pour inclure le code source
\usepackage{geometry}                     % pour les marges
\usepackage{algorithmic}                  % pour présenter les algos
\usepackage{algorithm}                    % pour présenter les algos
\usepackage{graphicx}                     % pour inclure des images
\usepackage{fancyvrb}                     % pour avoir du verbatim encadré
\usepackage{url}                          % pour inclure des URLs
\usepackage{geometry}                     % See geometry.pdf to learn the layout options. There are lots.
\geometry{a4paper}                        % ... or a4paper or a5paper or ... 
%\geometry{landscape}                     % Activate for for rotated page geometry
\usepackage[parfill]{parskip}            % Activate to begin paragraphs with an empty line rather than an indent
%\usepackage{amssymb}
%\usepackage{epstopdf}
\geometry{ hmargin=1.8cm, vmargin=2.5cm } % marges

% ------------------- TITRE ------------------------
\title{\includegraphics[width=100px]{heig-vd-logo.pdf} \\ \vspace{4cm}
\small{Laboratoire} \\ ASI : Applications et Services Internet \\ \vspace{2cm}
\huge{Serveur Web Multi-Threadé en Java} \vspace{2cm}} \author{Romain \bsc{de
Wolff} \\ IL2008 \\ \vspace{2cm} \\ Professeur M. Christian Buchs \vspace{2cm} }
\date{Lausanne, le 8 octobre 2007}  % Activate to display a given date or no date
\pagebreak{}
\begin{document}
\maketitle
% ---------------------- PAGE BREAK ---------------------------
\newpage
% ----------------- TABLE DES MATIERES ------------------------
\tableofcontents
\newpage
% ------------------- DEBUT DU CONTENU ------------------------

%%%%%%%%%%%%%%%%%%%%%%%%%%%%%%%%%%%%%%%%%%%%%%%%%%%%%%%%%%%%%%%%%%%%%%%%%%%%%%%
\section{Introduction}

Ce document contient les informations sur le déroulement du laboratoire
d'Application et Services Internet (\textbf{ASI}) dans le cadre des cours de la
formation d'ingénieur logiciel à la HEIG-VD. Ce laboratoire consiste à la
création d'un \textbf{serveur web en Java}. Un spécifieur/testeur (nous) et un
développeur (inconnu durant la première étape) allons devoir travailler
ensemble. \\

Dans une première partie, nous allons tout d'abord définir les démarches
effectuées, les spécifications et les cas-tests. Viendront ensuite les éléments
de la deuxième et dernière partie du rapport, qui comportera les informations
sur le développeur , des informations plus complète sur les différents tests
réalisés ainsi que des corrections. Nous répondrons aussi à quelques questions
techniques, comme savoir si le serveur établit des connexions persistantes ou
non.

%%%%%%%%%%%%%%%%%%%%%%%%%%%%%%%%%%%%%%%%%%%%%%%%%%%%%%%%%%%%%%%%%%%%%%%%%%%%%%%
\section{Démarche}
Afin de définir les spécifications, nous commençons par lire la données du
laboratoire une première fois. Ceci pour bien comprendre ce qu'on nous demande
et d'avoir une vue globale du projet. \\

L'étape suivante consiste à relire la donnée une deuxième fois muni d'un
marqueur afin de surligner les éléments importants. Nous cherchons tous les
éléments qui nous permettent de donner des directives factuelles au développeur.
Ceci afin de créer une liste de spécifications. Chacune de ces spécifications
devra être testée. C'est la deuxième étape qui consiste à dresser la liste des
\underline{cas-tests}. \\

Avant de faire la liste des cas-tests, nous avons relus les spécifications
trouvées et les avons comparée avec la RFC 1945. Puis la même chose avec le
support de cours de M. Buchs (slide de cours sur le \texttt{Protocole de
transfert HTTP}). Ceci afin de vérifier si les éléments correspondent, ainsi que
pour éviter des oublis. \\

Une fois la liste des spécifications terminée ainsi que la liste des cas-tests
correspondant, nous avons relus le tout afin de bien vérifier que tous les
éléments important y figurent, et que toutes les spécifications soient testées.

%%%%%%%%%%%%%%%%%%%%%%%%%%%%%%%%%%%%%%%%%%%%%%%%%%%%%%%%%%%%%%%%%%%%%%%%%%%%%%%
\section{Spécifications}

\begin{tabular}{ c | p{14cm} }
Numéro & Description \\
\hline
\hline
1  & Le serveur web est réalisé en \textbf{Java} et il est multi-threadé \\
2  & Il doit être capable de traiter plusieurs requêtes simultanément \\
3  & Il répond aux connexions faites à l'aide d'un navigateur web \\
4  & Des requêtes HTTP séparée sont envoyée pour chaque fichier\\
5  & Le serveur écoute sur un port fixe défini à l'aide de la ligne de commande
     lors du lancement du serveur. Le port spécifié doit être un \texttt{entier} 
     et il doit être supérieur à 1024.\\
6  & Quand le serveur reçoit une nouvelle demande de connexion, il établit une
     nouvelle connexion et sert la réponse dans une \emph{thread} séparée. \\
7  & Quand une erreur est rencontrée par le serveur, il renvoie un message 
     d'erreur approprié au format HTML \\
8  & Le serveur tourne indéfiniment, jusqu'à ce que la commande \texttt{CTRL+C}
     soit effectuée au clavier. \\
9  & Pour quitter le serveur, la commande \texttt{CTRL+C} est utilisée \\
10 & Chaque ligne de message de réponse du serveur se termine avec un retour à
     la ligne (\texttt{CR + LF}) \\
11 & Il doit gérer les exceptions TODO \\
12 & Le serveur affiche la ligne de statut ainsi que toutes les lignes d'en-tête 
     à l'écran tout en respectant les normes d'affichages vu en cours \\
13 & Uniquement la méthode \texttt{GET} est implémentée par le serveur (les autres
     requêtes sont traitées comme si c'est un \texttt{GET}) \\
14 & Le serveur renvoie les fichiers qui se trouve dans le même répertoire que lui
     au client, qu'on appelera la racine du serveur. Par exemple 
     \texttt{GET /index.html HTTP/1.1} \\
15 & Vérifier si le fichier demandé par le client existe ou non. Si il n'existe 
     pas, un message d'erreur doit apparaître chez le client. Le code de status
     \texttt{404} et le \emph{Reason-Phrase} \texttt{Not Found} doit être 
     retourné \\
16 & Un message de réponse contient 3 parties : la ligne de statut, les lignes 
     d'en-tête et le corps du message. La ligne de statut et les lignes d'en-tête
     sont terminé par un retour à la ligne (\texttt{CR + LF}). \\
17 & Le fichier d'erreur retourné si un fichier demandé n'existe pas est créé 
     par le serveur et il est renvoyé au client \\
18 & Le serveur renvoie les fichiers demandé par le client \\
19 & Le serveur doit déterminer les types de fichier qu'il renvoie en fonction 
     de son extension. \\
20 & Les extensions reconnues par le serveur sont les suivante : HTML, HTM, JPG, 
     JPEG et GIF.  Les types \textbf{MIME} correspondant retourné sont 
     \texttt{text/html}, \texttt{image/gif}, \texttt{image/jpeg} \\
21 & Si l'extension est inconnue, le serveur renvoie le type \textbf{MIME} suivant : 
     \texttt{application/octet-stream} \\
22 & Lorsque aucun fichier n'est demandé par le client, un fichier par défaut est
     affiché \\
23 & Si l'objet \texttt{/heig-vd.html} est demandé, effectuer une redirection
     standard HTTP 301 vers le site \texttt{http://www.heig-vd.ch} \\
24 & Le serveur doit mettre à disposition une adresse pour visualiser ses sources \\
25 & Le serveur doit être compatible avec la \textbf{RFC 1945}. \\

\hline
\end{tabular}

%%%%%%%%%%%%%%%%%%%%%%%%%%%%%%%%%%%%%%%%%%%%%%%%%%%%%%%%%%%%%%%%%%%%%%%%%%%%%%%
\section{Cas-tests}

\begin{tabular}{ c | c | p{7cm} | p{6cm} | c }
CT & Sp & Procédure & Résultat attentu & OK\\
\hline
\hline
1 & 1 & Ouvrir la ligne de commande et lancer le serveur à l'aide de la commande 
        \texttt{java nomServeur <port>}. Ouvrir 2 (ou plusieurs) navigateur et accéder
        à l'adresse \texttt{http://localhost:<port>}.  
        & Le serveur doit se lancer & \\
2 & 17 & Ouvrir le navigateur et entrer le nom d'un fichier pas présent sur le
         serveur. On peut par exemple
         choisir le nom \texttt{fichierInexistant<datefutur>.html} 
         & Le serveur doit afficher un message au format HTML. Vérifier qu'il 
         n'existe pas de fichier HTML dans le dossier du serveur qui contient le 
         texte affiché à l'écran. \\
         
        
\hline
\end{tabular}

%%%%%%%%%%%%%%%%%%%%%%%%%%%%%%%%%%%%%%%%%%%%%%%%%%%%%%%%%%%%%%%%%%%%%%%%%%%%%%%
\section{Conclusion}


%%%%%%%%%%%%%%%%%%%%%%%%%%%%%%%%%%%%%%%%%%%%%%%%%%%%%%%%%%%%%%%%%%%%%%%%%%%%%%%	
\end{document}  
