
% -----------------------------------------------------------------------------
% Auteurs : Romain de Wolff, Bruno da Silva, Gilles Meier
% -----------------------------------------------------------------------------
\documentclass[10pt,a4paper,titlepage]{article}
\usepackage[utf8]{inputenc}     % encodage des characteres en utf8
\usepackage[francais]{babel} % pour la table des matières en français 

\usepackage{url} % pour les liens internet
\usepackage[colorlinks=true,linkcolor=black,bookmarks=true,bookmarksopen=true]{hyperref} % rendre les liens clickable
\usepackage{fancyhdr}	 
\usepackage{listings} 
\lstset{language=Java, breaklines, fontadjust, inputencoding=utf8, basicstyle=\small, numbers=left, numberstyle=\tiny, tabsize=2}

\usepackage[dvips]{graphicx}
\usepackage{epstopdf}
\usepackage[parfill]{parskip}             % Activate to begin paragraphs with an empty line rather than an indent


% -----------------------------------------------------------------------------
%   Info sur le labo
% -----------------------------------------------------------------------------
\newcommand{\branchetag}{ASI}
\newcommand{\branche}{Applications et Services Internet}
% \newcommand{\labonummer}{}
\newcommand{\laboname}{SSL - JSSE (Java Secure Socket Extension)}
\newcommand{\auteurOne}{Romain de Wolff}
% \newcommand{\auteurTwo}{Bruno da Silva}
\newcommand{\promo}{IL2008}
\newcommand{\titreDocument}{Rapport de laboratoire}

% -----------------------------------------------------------------------------


% -----------------------------------------------------------------------------
% Pour l'utilisation de code
% -----------------------------------------------------------------------------

\usepackage{listings} 
%\lstset{language=Java, breaklines, fontadjust, inputencoding=utf8, basicstyle=\small, numbers=left, numberstyle=\tiny, tabsize=2}

\usepackage{courier}
\usepackage{color}

% color definitions
\definecolor{dkgreen}{rgb}{0,0.6,0}
\definecolor{gray}{rgb}{0.5,0.5,0.5}
\definecolor{lightblue}{rgb}{0.92,0.92,1}

\lstset{language=Html,
  %keywords={break,case,catch,continue,else,elseif,end,for,function,
  %   global,if,otherwise,persistent,return,switch,try,while},
  keywords={script, document, function},
  basicstyle=\ttfamily\small,
  % basicstyle=\scriptsize,
  keywordstyle=\color{blue},
  commentstyle=\color{dkgreen},
  stringstyle=\color{red},
  numbers=none,
  numberstyle=\tiny\color{gray},
  stepnumber=1,
  numbersep=10pt,
  backgroundcolor=\color{lightblue},
  tabsize=2,
  linewidth=0pt,
  showspaces=false,
  showstringspaces=false,
  frame=single,
  framexleftmargin=10pt,
  framexrightmargin=10pt,
  framexbottommargin=7pt,
  framextopmargin=7pt,
  linewidth=335pt, % largeur de la ligne de code affichée
  xleftmargin=10pt, % espace avant le debut du cadre
  aboveskip=20pt
}

% -----------------------------------------------------------------------------
% En tete et pied de page

\pagestyle{fancy} % defini nos propre header & footer
\fancyhf{} % delete current header and footer 
\fancyhead[L]{\branchetag}
\fancyhead[C]{\laboname}
\fancyhead[R]{\auteurOne} 
\fancyfoot[L]{\includegraphics[width=3cm]{img/HEIG-VD.jpg}}
\fancyfoot[R]{\bfseries\thepage}

\renewcommand{\headrulewidth}{0.5pt} 
\renewcommand{\footrulewidth}{0.1pt} 
\addtolength{\footskip}{10.0pt} % space for the rule 
\fancypagestyle{plain}{
	\fancyhead{} % get rid of headers on plain pages 
	\fancyfoot{}
	\renewcommand{\headrulewidth}{0pt} % and the line 
	\renewcommand{\footrulewidth}{0pt} % and the line 
}

\author{\auteurOne}
\title{\branchetag : \laboname}
\date{\today}

\begin{document}

% -----------------------------------------------------------------------------
% Page de titre
\pagenumbering{Roman}
\pagestyle{headings}
\begin{titlepage}
	\begin{center}

	\includegraphics[width=6cm]{img/HEIG-VD.jpg}
	
		\vspace{3cm}
		\LARGE \branche %Laboratoire No %\labonummer \\
		\vspace{3cm}\\
		\Huge \laboname \\
		\vspace{3cm}

		\Large \textsc{\titreDocument} \\
		\vspace{3cm}

		\large \auteurOne \\
		% \auteurTwo \\ % pour un eventuelle deuxième auteur
		\vspace{10pt}
		\normalsize \textsc{\promo} \\
		\vspace{1cm}
		\today
	\end{center}
\end{titlepage}
% -----------------------------------------------------------------------------

% -----------------------------------------------------------------------------
% Table des matières
\tableofcontents
\newpage
\pagestyle{fancy}
\pagenumbering{arabic}
 
% -----------------------------------------------------------------------------
\section{Introduction}
% -----------------------------------------------------------------------------

% -----------------------------------------------------------------------------
\section{Clé publique générée}
% -----------------------------------------------------------------------------

Voici la clé publique qui nous a été generée par Keytool :

\scriptsize{
\begin{verbatim}
-----BEGIN NEW CERTIFICATE REQUEST-----
MIIBujCCASMCAQAwejELMAkGA1UEBhMCQ0gxCzAJBgNVBAgTAlZEMREwDwYDVQQHEwhMYXVzYW5u
ZTEZMBcGA1UEChMQd3d3LlRBR0FEQVJULmNvbTEcMBoGA1UECxMTUGVyc29uYWwgV2ViIFNlcnZl
cjESMBAGA1UEAxMJMTI3LjAuMC4xMIGfMA0GCSqGSIb3DQEBAQUAA4GNADCBiQKBgQCjcLrCVl/h
50CuSHjNevhTrRS0bCQ1oCN27c3hTLdDbLVjDNqUJqziTXpowFTUXmM/hrbKwVzM5+I4krwx/6dW
oVVhaGywxkQwN4mQ2rgFvkdm8xIpPKfyVMTLYRQLfd89qLYC8C0SUR3MqzuNRpT7lnla1RB9A6Mg
IIx53mf8UQIDAQABoAAwDQYJKoZIhvcNAQEEBQADgYEAAht/hvwHdT1Qb5ZPe6EmBbMJe6VozqQT
yzaA2q6+4Y+FzuQ0PT7oePyg22e6HTiEtxRIhNGiCXVlceeNxKYFBGoBGVSGOHvauWFGRntErntQ
X7vYKW5XCjHfEpsMwKsj42b4zMFn743IT/LmiC/NsghW3q+UD7AUslld4+XaX68=
-----END NEW CERTIFICATE REQUEST-----
\end{verbatim}
}

% -----------------------------------------------------------------------------
\section{Réponses aux questions}
% -----------------------------------------------------------------------------

Lors de la connexion sur notre serveur HTTPS à l'aide du navigateur Firefox, le serveur nous affiche une alerte comme le montre la figure \ref{fig:connex}

\begin{figure}[htbp]
   \begin{center}
      \includegraphics[width=345px]{img/1.png}
   \end{center}
   \caption{Alerte affichée lors de la connexion sur le site sécurisé.}
	\label{fig:connex}
\end{figure}

Nous acceptons ce certificat et nous allons voir le site s'afficher. On remarque que le site est sécurisé grâce à l'icone représentant un cadenas (an bas à droite dans Firefox) que l'on peut voire sur la figure \ref{fig:icone}.

\begin{figure}[htbp]
   \begin{center}
      \includegraphics[height=23px]{img/2.png}
   \end{center}
   \caption{Icone dans la barre des tâches du navigateur Firefox.}
	\label{fig:icone}
\end{figure}

En cliquant sur le cadenas on peut afficher les informations relatives à la sécurité et donc du certificat que l'on a accepté. La figure \ref{fig:infosecu} nous montre à quoi ressemble cette fenêtre.

\begin{figure}[htbp]
   \begin{center}
      \includegraphics[width=345px]{img/3.png}
   \end{center}
   \caption{Information sur la sécurité du site.}
	\label{fig:infosecu}
\end{figure}

La figure \ref{fig:infosecu} nous montre les informations sur le certificat et nous dit que nous faisons confiance au CA mentionné. En cliquant sur le bouton “View” on obtient les informations sur le certificat, exactement les information que nous avons introduites lors de la création à l'aide de \emph{Keytool}. La figure \ref{fig:certif} nous montre cette fenêtre.

\begin{figure}[htbp]
   \begin{center}
      \includegraphics[width=345px]{img/4.png}
   \end{center}
   \caption{Affichage détaillée des informations sur le certificats que nous avons créé.}
	\label{fig:certif}
\end{figure}

\subsection{En examinant le certificat du serveur, comment pouvez-vous en déduire que celui-ci est auto-signé?}

On le voit bien sur la figure \ref{fig:certif}: les champs “Issued To” et “Issued By” sont identiques. On sait dès lors que le certificat est auto-signé. 

\subsection{Pourquoi serait-il intéressant de faire signer le certificat par une entité comme Verisign?}

Verisign est une entreprise de type CA: elle émet des certificats qu'elle vends. Ces certificats sont réputés fiable. L'avantage d'avoir un certificat d'un CA reconnue est que les utilisateurs qui se connecte sur le site peuvent, grâce à la renomée de Verisign, savoir que le site utilise une encryption de qualité. C'est donc pour des questions de sécurité et de véracité que nous avons avantage à utiliser les services offert par une société comme Verisign.

L'utilisateur sera donc mis en confiance et n'aura plus d'avertissement du navigateur comme quoi le certificat est douteux. 

% -----------------------------------------------------------------------------
\section{Conclusion}
% -----------------------------------------------------------------------------



% -----------------------------------------------------------------------------
\section{Références}
% -----------------------------------------------------------------------------

\small
\begin{description}
	\item[\url{http://jquery.com/}] {Site officiel de jQuery.}
\end{description}
\end{document}