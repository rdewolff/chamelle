%==============================================================================
% document template for a LaTeX file
% Created : 22.11.2006 by Romain de Wolff
% Modif : 8 octobre 2007, respect standart HEIG-VD pour travaux de diplomes
%==============================================================================
% configurations du document
\documentclass[a4paper, 11pt]{article}
\usepackage[utf8]{inputenc}
\usepackage[cyr]{aeguill}
\usepackage[frenchb]{babel}               % \og et \fg pour les guillemets
\usepackage{url}                          % pour inclure des URLs
\usepackage{color}                        % utilise pour le code source
\usepackage{listings}                     % pour inclure le code source
\usepackage{geometry}                     % pour les marges
\usepackage{algorithmic}                  % pour presenter les algos
\usepackage{algorithm}                    % pour presenter les algos
\usepackage{graphicx}                     % pour inclure des images
\usepackage{fancyvrb}                     % pour avoir du verbatim encadre
\usepackage{url}                          % pour inclure des URLs
\usepackage{geometry}                     % See geometry.pdf to learn the layout options
\usepackage{moreverb}					  % permet d'encadrer le verbatim a l'aide de la commande \begin{boxedverbatim}
\usepackage{longtable}
% afin de créer des jolies listing de code
\usepackage{courier}
\usepackage{color}

% color definitions
\definecolor{dkgreen}{rgb}{0,0.6,0}
\definecolor{gray}{rgb}{0.5,0.5,0.5}
\definecolor{lightblue}{rgb}{0.92,0.92,1}

\lstset{language=Java,
  %keywords={break,case,catch,continue,else,elseif,end,for,function,
  %   global,if,otherwise,persistent,return,switch,try,while},
  % basicstyle=\ttfamily\small,
  basicstyle=\scriptsize, 
  keywordstyle=\color{blue},
  commentstyle=\color{dkgreen},
  stringstyle=\color{red},
  numbers=none,
  numberstyle=\tiny\color{gray},
  stepnumber=1,
  numbersep=10pt,
  backgroundcolor=\color{lightblue},
  tabsize=2,
  linewidth=0pt,
  showspaces=false,
  showstringspaces=false,
  frame=single,
  framexleftmargin=10pt,
  framexrightmargin=10pt,
  framexbottommargin=7pt,
  framextopmargin=7pt,
  linewidth=430pt, % largeur de la ligne de code affichée
  xleftmargin=10pt, % espace avant le debut du cadre
  aboveskip=20pt
}

\geometry{a4paper}                        % ... or a4paper or a5paper or ... 
%\geometry{landscape}                     % Activate for for rotated page geometry
\usepackage[parfill]{parskip}             % Activate to begin paragraphs with an empty line rather than an indent
%\usepackage{amssymb}
%\usepackage{epstopdf}                    % Pour creer des references divers (webographie, etc) 
\usepackage{bibtopic}
\geometry{ hmargin=2.8cm, vmargin=2.5cm } % marges
%==============================================================================
% debut macro 
% permet de mettre en gros la premiere lettre d'un paragraphe
\font\capfont=cmbx12 at 44.87 pt % or yinit, or...?
\newbox\capbox \newcount\capl \def\a{A}
\def\docappar{\medbreak\noindent\setbox\capbox\hbox{%
\capfont\a\hskip0.15em}\hangindent=\wd\capbox%
\capl=\ht\capbox\divide\capl by\baselineskip\advance\capl by1%
\hangafter=-\capl%
\hbox{\vbox to8pt{\hbox to0pt{\hss\box\capbox}\vss}}}
\def\cappar{\afterassignment\docappar\noexpand\let\a }
% fin macro
%==============================================================================

%==============================================================================
% premiere page
\title{ %\includegraphics[width=100px]{HEIG-VD.jpg} \\ \vspace{4cm}
\small{RIM - Recherche d'Information Multimedia} \\ \vspace{2cm}
\huge{Labo 6} \\ \vspace{1cm} 
Analyse des hyperliens\\
\small{Couplage bibliographique, Co-Citation, Hub, Autorité et PageRank}} 
\vspace{2cm}
\author{Romain \bsc{de Wolff} et Simon \bsc{Hintermann}\\ IL2008 \\ \vspace{2cm} \\ Professeurs Mme. Laura Elena Raileanu \\ et Mme Nastaran Fatemi \vspace{2cm} 
}
\date{Lausanne, le 5 decembre 2007}  % Activate to display a given date or no date
\pagebreak{}
\begin{document}
\maketitle
\thispagestyle{empty} % enleve le numero de page sur la page de titre (uniquement)
\newpage
%==============================================================================
% configuration des numerotations de pages (chiffres romains)
\pagenumbering{arabic} \setcounter{page}{1} 

% Configuration de la distance d'interligne 
{\setlength{\baselineskip}{1.2\baselineskip}}
% Configureation de la distance entre les paragraphes
\parskip=12pt
%==============================================================================
%==============================================================================
\section{Objectifs} 
%==============================================================================

Le but de ce laboratoire est d'implémenter un système permettant
d'effectuer les différentes méthodes de calculs des hyperliens. Nous allons
tout d'abord tester ces calculs sur un graphe donné puis implémenter ces
calculs dans notre moteur de recherche.

Les méthode de calculs utilisées sont les suivantes: Couplage Bibliographique, 
Co-Citation, Hub, Autorité et PageRank. 

Ces méthodes de calculs sont décrites dans les support de cours ainsi que dans 
la donnée du laboratoire.

%==============================================================================
\section{Demarches adoptées} 
%==============================================================================

Dans ce chapître, nous décrivons les démarches que nous avons effectués durant
ce laboratoire.

\subsection{Implémenter les méthodes de calculs}

Après avoir parcourus le code mis à disposition, nous avons determiné
quel méthodes de calculs nous allions avoir besoin afin de calculer les
différentes analyses d'hyperliens. 

Nous avons créé une classe \texttt{Matrix} qui met à disposition des méthodes
static afin d'effectuer ces différents calculs.

Il s'agit des méthodes suivantes : 

\begin{description}
\item[AdjacencyMatrix transpose(AdjacencyMatrix m)] retourne la transposée
d'une matrice passée en paramètre
\item[multiply(AdjacencyMatrix m, Vector<Double> ac)] multiplie
une matrice et un vecteur (de la même taille), retourne un vecteur
\item[multiply(AdjacencyMatrix m1, AdjacencyMatrix m2)] multiplie deux matrices
et renvoie la matrice résultante
\item[multiply(Vector<Double> v1, Vector<Double> v2)] multiplie deux vecteurs
(contenant des \texttt{doubles}) et renvoie le vecteur résultant
\item[add(Vector<Double> v1, Vector<Double> v2)] additionne deux vecteurs
(contenant des \texttt{double}) et renvoie le vecteur résultant
\end{description}

Une fois ces méthodes de calculs effectuées et testées, nous pouvons implémenter
les méthodes d'analyze des hyperliens qui se trouvent dans le fichier
\texttt{LinkAnalysis.java}.

On note que la méthode \texttt{calculateAc()} utilise la méthode
\texttt{calculateHc()} en changeant uniquement le paramètre de la matrice
passée en paramètre. En effet, le calcul est le même, sauf que la matrice
utilisée doit être transposée auparavant. C'est ce que nous faisons.

Pour pouvoir répondre et analyser le graphe fourni, il nous faut le représenter
au format texte. Voici cette représentation:

%\begin{lstlisting}
\begin{verbatim}
# nodes
1
2
4
6
8
9
10
20
22
25
40
41
42
49
60
80
93
100
# edges (16)
1;10
2;20
4;2;
4;40
4;41
4;42
6;60
8;4
8;80
9;1
9;10
10;100
22;20
25;2
25;20
49;4
\end{verbatim}
% \end{lstlisting}


Nous pouvons tester notre travail. Pour cela nous allons créer notre programme
principal de test des analyzes d'hyperliens: \texttt{Labo.java} dans le package
\texttt{rim.analze}. Nous allons utiliser la classe \texttt{GraphFileReader}
afin de lire le contenu du fichier représentant notre graphe. Une fois la
matrice d'adjascence récuperée, nous effectuons les calculs et affichons les
résultats obtenus à l'écran. 

Voici le résultat des calculs effectués sur le graphe complet donné dans la
donnée:

\begin{verbatim}
------------------------------------------------
- RIM - Laboratoire 6                          -
- Auteurs : Romain de Wolff & Simon Hintermann -
------------------------------------------------
BC 25 <-> 22 : 1.0
CC 20 <->  2 : 1.0
------------------------------------------------
- Node -  Hubs   - Authority - PageRank        -
------------------------------------------------
    1    0,08018   0,10073     0,03089
    2    0,18040   0,57918     0,10928
    4    0,82181   0,16368     0,04918
    6    0,01002   0,00000     0,01259
    8    0,13029   0,00000     0,01259
    9    0,13029   0,00000     0,01259
   10    0,01002   0,16368     0,07923
   20    0,00000   0,54141     0,18757
   22    0,18040   0,00000     0,01259
   25    0,46102   0,00000     0,01259
   40    0,00000   0,31477     0,09098
   41    0,00000   0,31477     0,09098
   42    0,00000   0,31477     0,09098
   49    0,08018   0,00000     0,01259
   60    0,00000   0,01259     0,03089
   80    0,00000   0,10073     0,03089
   93    0,00000   0,00000     0,01259
  100    0,00000   0,01259     0,12099
\end{verbatim}

\subsection{Intégration dans le moteur de recherche}

Sachant que notre version du laboratoire précédant comporte encore quelques
erreurs, nous avons utilisé celle d'une de nos collègues. Laurent Prévost et
Jonas Schmid ont eut la gentillesse de nous donner leurs sources du laboratoire
précédant.

Nous avons commencer par prendre connaissance de leur code et effectuer quelques
adaptations pour que celui-ci fonctionne sur nos système d'exploitation.
\texttt{*nix}.

Pour cela nous avons tout d'abord modifier le chemin des
\textit{common words} configuré dans le fichier \texttt{Constants.java}. 

%==============================================================================
\section{Conclusion personnelle} 
%==============================================================================

Comme au laboratoire précédant, il est difficile de faire des tests sur un site
web comportant autant de page que celui de la HEIG-VD. Nous avons donc fait des
tests sur d'autre site compotant moins de pages, comme le site
\url{http://www.kdanse.ch}.



%==============================================================================
\end{document}  
